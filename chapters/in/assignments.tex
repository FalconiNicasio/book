% chktex-file 8

\chapter{Assignments}
\label{c:assignments}

\FILENAME\

The assignments are listed in chronological order. All assignments
posted here are supposed to be conducted by you. There is a slight
delay between assignments posted in piazza and the assignments in this
section as TAs need to be integrating them. Typical delay is one
business day. Business days are defined as Mon-Fri 9am-5pm
EST. Thus, it is often better for those working on the weekends to
visit piazza.com instead. The assignments are ordered by class. Please
focus on conducting the assignments listed either here on in
piazza. Just as any textbook has many exercises, we are providing
selected exercises for you as assignments. Not every exercise
mentioned in a chapter has to be done.

\begin{IU}
  There can not be any confusion which assignments have been issued,
  as they are all pinned in piazza and you can visit piazza.com to
  find out which are assigned and listed there.
\end{IU}

If you use a calendar system, it is in your responsibility
to manage it in such a system. This could include Google, Outlook,
CANVAS, and may others. Naturally you can also use to do lists that
you can manage as part of your github repository issues, once you have
access to. This is probably the preferred method as it allows you to
add the tasks yourself and you can check them on and off as you have
conducted the assignments. 

This way we teach you how to use open source technologies to
coordinate your own work with your own time management tools and
constraints in mind. This is in contrast to just using CANVAS as
CANVAS does not support open source developments in teams. Furthermore
it is unlikely that you will ever use CANVAS after you graduate. We
rather like to have you use systems that you use after you graduate.
However, if you still like to use CANVAS for alerting you, that is
entirely up to you as you can add assignments yourself to CANVAS. 

\smallskip

\begin{IU}
  All assignments are due Monday morning at 9:00 am est. No exceptions
  unless otherwise specified.
\end{IU}

Learning outcome:
\begin{itemize}
\item You can manage assignments with any system you like
\item You can manage your own events in CANVAS and continue CANVAS to
  notify you of events. You are responsible of entering these events
  in CANVAS.
\item A better way of managing your tasks is to use github issues.
\item Alternatively you can use your own calendar system
\end{itemize}


\section{Assignments E222}
\label{s:e222-assignment}
\label{s:e222-assignments}
\index{Assignments!E222}

All assignments posted here are supposed to be conducted by you. 

\subsection{Bio Post}
\label{E:e222-bio}

\begin{exercise}\label{E:e222-bio-piazza}
{\bf Bio Post on Piazza.} Please post a formal bio to piazza
\end{exercise}


\begin{exercise}\label{E:e222-bio-googledocs}

  {\bf Bio Post in Google doc.} due Fed 5, 2018. 
  
  After you have posted it to piazza
  copy your updated formal bios into the following document.  Make
  sure you use 3rd person and stay formal. This is a formal
  bio. Comment on the effectiveness of using the cloud service for
  this task. A the end of the document. This assignment does not
  replace the post of the bio to piazza, it is used to gather all bios
  in one document and to evaluate if google docs is a good tool for
  this kind of task. Remember we have lots of students and google is
  used often just with small groups.
 
 \smallskip

 {\hfill
   \href{https://docs.google.com/document/d/1pNK94qoRfZkill_JrGAzjd8aQ6Aar0pEXhU_Tgog0W0/edit?usp=sharing}{E222
     Link to google doc $\mapsto$}}

\end{exercise}

Learning Outcomes:
\begin{itemize}
\item This is a cloud class that will use a number of cloud based
  systems. You can by using them identify differences, advantages, and
  disadvantages.
\item in contrast to IU services which typically are done via SSO the
  community cloud services provide you with many different accounts,
  you will be needing to find a system on how to manage these
  accounts.
\item Just like in a class the professor communicates to all students
  and encourages a dialog. Piazza simulates this isn some form for
  online students. It is important not to just read the piazza
  e-mails, but to actually visit every post you get by using the click
  here link and inspect the post. The answer from students to
  questions such as do i need to read all questions is the same as in
  a live lecture, do I need to listen to the instructor? I am sure you
  know the answer.
\end{itemize}


\subsection{Cloud Accounts}
\label{E:e222-iu-google-services}

\begin{exercise}\label{E:e222-iu-google}

  {\bf IU Google Services.} due Feb 5, 2018
  
  This assignment is only for those that do
  not yet have access to our google documents This assignment does not
  have to be conducted for anyone that has access to our google
  documents for bios, and the technology list

  \begin{itemize}
 
  \item What is the difference between umail.iu.edu and iu.edu? Tip:
    the answer is provided in the IU knowledge base

  \item Login via the iu.edu account and not the umail.iu.edu account
    to google and open the document for the bio. Paste the bio into
    the document.

  \item Explain why IU has two different google services and
    logins. As we use cloud in this class, it is important to
    understand this and what implication this has. This is not just an
    assignment to give you access to the service, but to make you
    think why this works like this.

  \item Can you imagine a different way this ought to work?

  \end{itemize}

\end{exercise}

Learning Outcome:

\begin{itemize}
\item IU has multiple account names and e-mail addresses for different
  services. Also service names are under transition and we assume
  umail will be removed. This assignment will provide you with
  identifying which iu e-mail is used for which service. Do not
  hesitate to contact UITS for help if you can not access some
  services. Find first answers in IU Knowledge base.
\item Use important cloud services
\end{itemize}


\subsection{Account Creation}

\begin{exercise}

  {\bf Account Creation: github.com}
  
  If you do not have a github.com
  account, go to github.com and apply for a \url{https://github.com}
  account. Write down your account name and remember the password. You
  will need the account for upcoming assignments.

\end{exercise}

\begin{exercise}

  {\bf Account Creation: futuresystems.org}
  
  If you do not have a futuresystems.org account go to
  \url{https://portal.futuresystems.org/user/register} and apply for an
  account. Write down your account name and remember the password. You
  will need the account for upcoming assignments.

\end{exercise}

\begin{exercise}
  {\bf Account Creation: chameleoncloud.org}. 
  
  If you do not have a chameleon cloud account please go to 
  \url{https://www.chameleoncloud.org} and apply for an
  account only. Do not apply for a project. Write down your account
  name and remember the password. You will need the account for
  upcoming assignments.
\end{exercise}

\begin{exercise}
   {\bf Account Collection Form } due Fed 5, 2018
 
 Fill out the form so we can activate your accounts. You will need the account for upcoming assignments.
 
  {\hfill \href{https://goo.gl/forms/W0MdgoJoY8F6Vt9Q2}{E222 Account Collection Form $\mapsto$}}
 

\end{exercise}

Learning Outcome:
\begin{itemize}
\item This class uses cloud computing resources. THe resources listed
  are the onc e we use. Most importantly you will get a github
  repository created that you will be using for the class at
  \url{https://github.com/cloudmesh-community}. We can only create
  your repository if we know your github username. Please note the
  repository is viewable. As we are working as prat of this class as
  open source project it is natural that all work is done in the open.
\end{itemize}


\subsection{Entry Survey}
\begin{exercise}
    {\bf Entry Survey}
    
 Please fill out the following survey ASAP as it will determine some
 of the class material we prepare based on your feedback. The survey
 is really simple and can be finished in under 5
 minutes. \url{https://goo.gl/forms/Q04FW9eBM7eyL0Lv1}

\end{exercise}

Learning Outcome:
\begin{itemize}
\item We ned to identify what background you have and which resources
  we have as to potentially adapt our class material, but also to
  identify which resources you need to use in addition to your own.
\end{itemize}


\subsection{Account Setup}

\begin{exercise} {\bf Github Setup} due Feb, 2018

This exercise will assist you with your github account setup. Please
find the README.yml file in the link below and copy it into your
github repository and fill out with your
information. \url{https://github.com/cloudmesh-community/hid-sample/blob/master/README.yml}
Remove the README.md file. After your \verb|.yml| file is correctly setup you
now need to put your bio in your repository, naming it like so
\verb|bio-lastname-firstname.tex| where lastname is your lastname and so
on. Put the \verb|.tex| file in your repository and use the latex function
subsection\{lastname,firstname\} followed by the text of your bio.

\end{exercise}

Learning Outcome:
\begin{itemize}
\item Setting up your github repository. Letting us know who you
  are. Filling out a form
\end{itemize}


\subsection{Eve REST Service}
\label{E:rest-eve}
\begin{exercise}
Read the Sections Overview of REST (Section~\ref{c:rest} and
Eve~\ref{s:eve-intro}).

\end{exercise}

Learning Outcome:
\begin{itemize}
\item Learn how to create a simple REST service. This assignment will
  be expanded upon throughout the semester. Make sure you have a solid
  understanding of it.
\end{itemize}

\begin{exercise}
Assignment catch up: please look at all previous assignments and do
them. 

\end{exercise}

Learning Outcome:
\begin{itemize}
\item It is not good in this class to procrastinate.
\end{itemize}

\begin{exercise}

AI cloud service: This assignment will evolve, therefore we just
provide the link to it in piazza. Please visit it:

\url{https://piazza.com/class/jc9dcfnbi045kv?cid=34}

\end{exercise}

Learning Outcome:
\begin{itemize}
\item You will be combining various technologies and practices that
  you learned in class as part of a project assignment.
\item You wil be exposed to various AI technologies and algorithms
  that you can explore further by integrating them in your project.
\end{itemize}


\section{Assignments E516, I524, E616}
\label{s:616-assignments}
\index{Assignments!E516}
\index{Assignments!E616}
\index{Assignments!I524}

\subsection{Bio Post}\label{a:616-bio}

\begin{exercise}\label{E:616-bio-piazza}
{\bf Bio Post on Piazza.} Please post a formal bio to piazza
\end{exercise}

\begin{exercise}\label{E:616-bio-googledocs}

  {\bf Bio Post in Google doc.} due Feb 5, 2018
  
  After you have posted it to piazza
  copy your updated formal bios into the following document.  Make
  sure you use 3rd person and stay formal. This is a formal
  bio. Comment on the effectiveness of using the cloud service for
  this task. A the end of the document. This assignment does not
  replace the post of the bio to piazza, it is used to gather all bios
  in one document and to evaluate if google docs is a good tool for
  this kind of task. Remember we have lots of students and google is
  used often just with small groups.
 
 \smallskip

 {\hfill
   \href{https://docs.google.com/document/d/1ejzlKYqC3dLac8WXVpcPQsJh1j4BDqRxxgGg1cFQbeQ/edit?usp=sharing}{E516
     Link to google doc $\mapsto$}}


 \end{exercise}

Learning Outcomes:
\begin{itemize}
\item This is a cloud class that will use a number of cloud based
  systems. You can by using them identify differences, advantages, and
  disadvantages.
\item in contrast to IU services which typically are done via SSO the
  community cloud services provide you with many different accounts,
  you will be needing to find a system on how to manage these
  accounts.
\item As we want to avoid plagiarism within the class, IU google doc
  is used to identify and assign unique assignments to the
  students. Therefore you need to have access to IU google. YOu can
  not use your regular google ID. You will need to learn how to manage
  multiple google IDs if you also have a non IU google ID set up.
\item Just like in a class the professor communicates to all students
  and encourages a dialog. Piazza simulates this isn some form for
  online students. It is important not to just read the piazza
  e-mails, but to actually visit every post you get by using the click
  here link and inspect the post. The answer from students to
  questions such as do i need to read all questions is the same as in
  a live lecture, do I need to listen to the instructor? I am sure you
  know the answer.
\end{itemize}

\subsection{IU Google Services}
\label{E:e616-iu-google-services}

\begin{exercise}\label{E:616-iu-google}

  {\bf IU Google Services:} due Feb 5, 2018
  
  This assignment is only for those that do
  not yet have access to our Google documents This assignment does not
  have to be conducted for anyone that has access to our Google
  documents for bios, and the technology list

  \begin{itemize}
 
  \item What is the difference between umail.iu.edu and iu.edu? Tip:
    the answer is provided in the IU knowledge base

  \item Login via the iu.edu account and not the umail.iu.edu account
    to google and open the document for the bio. Paste the bio into
    the document.

  \item Explain why IU has two different google services and
    logins. As we use cloud in this class, it is important to
    understand this and what implication this has. This is not just an
    assignment to give you access to the service, but to make you
    think why this works like this.

  \item Can you imagine a different way this ought to work?

  \end{itemize}

\end{exercise}

Learning Outcome:

\begin{itemize}
\item Procrastination in this class does not pay off. 
\item This assignment
  was added for those students that did not do it in the first week. If you
  procrastinate, you will see that you have additional work to do. The
  administrative tasks must not be late as we need to set up and
  coordinate with you and if not done in time you lose valuable time
  for the class. 
\item Please note that we have very few assignments, but that they
  need to be done by the deadline. Yu need to start early as they take
  multiple weeks to complete.
\end{itemize}

\subsection{Big Data Collaboration}
\label{E:616-bigdata-collab}

\begin{exercise}\label{E:616-big-data-and-collaboration} {\bf Big
    data and collaboration.} due Feb 2018
    
  The purpose of this assignment is
  manifold; test the ability of Google docs to be used in
  collaborative fashion by more than a small group and report on the
  experience. Good Things and bad things, learn on how to use Google
  docs with headings and table of contents learn how to gather
  resources quickly with hyperlinks to web resources or articles and
  translate them into formal academic references. Most importantly
  convey some very important feature of big data.Contribute this into
  the handbook for everyone's benefit (done by TAs).  \smallskip

  \noindent {\bf Task:} Your task is to identify Big Data size related
  articles and Web resources and produce a historical development of
  the growth of this data

  {\hfill
    \href{https://docs.google.com/document/d/1ZHNdhX_Jx7uBQo0kthSYQ6TQR8_KNbgOwH2EuqBQcjY/edit?usp=sharing}{E516
      Link to google doc $\mapsto$}}


\end{exercise}

Learning Outcome:

\begin{itemize}
\item Learn about good and bad things in google docs
\item Use google docs for collaborative writing 
\end{itemize}


\subsection{New Technology List}
\label{E:616-new-tech}

\begin{exercise} {\bf Technology List} due Jan 29, 2018

 The handbook contains a large number of technologies to which an
 abstract is provided. Your task is to identify FIRST not to do an abstract but to
 collaboratively gather a LIST of new technologies that are important
 in Cloud and Big Data. We suggest doing this in a google docs document
 first. Write Lastname, Firstname, class id behind the technology so we
 know who contributed it. Indicate also if commercial, or open source,
 We are mostly interested in open source activities. Keep the list
 sorted by alphabet. Use a bullet so formatting is preserved

\smallskip

{\hfill \href{https://docs.google.com/document/d/1LeHGHTSBbaPXYVor0efhmi5W7JJjS7EQHABHqgRAPuU/edit?usp=sharing}{New Technology List $\mapsto$}}

\smallskip

Example: OpenWhisk, \url{https://openwhisk.apache.org/}, open source, Gregor von Laszewski, e616

\end{exercise}

Learning Outcome:

\begin{itemize}
\item Get an overview about technologies important for the class.
\item Learn how to explore on your own account.
\item Identify good and bad technology abstracts.
\item Identify a unique technology that you explore further.
\end{itemize}



\subsection{New Technology Abstract}
\label{E:616-new-tech-abstract}

\begin{exercise} {\bf Technology Abstract} due Feb 5, 2018

We have gathered with the technology list
\url{https://piazza.com/class/jbkvbp3ed3m2ez?cid=50} a number of
technologies that are not yet covered in the handbook or
need improvement in the handbook.


The TAs will be selecting about 5 technologies for each student. Each
student will write high-quality non-plagiarized abstracts which bibtex
references.
 

\end{exercise}

Learning outcomes:

\begin{itemize}

\item Identify how to not plagiarize
\item Work in a large team (with coordination by TAs)
\item Use bibtex and jabref for reference management which you will be
  using for your final paper
\item Find new trends in big data and cloud computing
\item Learn how to write meaningful abstracts. Abstracts are used
  often to provide an overview of a technology. It is important to
  distinguish between an objective abstract and a paragraph that
  contains advertisement potentially written by the authors or others
  about a product. Terms such as easy to use, best, should be
  quantified with references, not just citations that mention they are
  easy to use, but potentially a statistics or a user survey if you
  really like to use these terms. 
\item If possible engage in discussions with other students. Identify
  a colleague and discuss your abstracts with the colleague. Improve
  each others abstracts.
\end{itemize}

\begin{exercise}{\bf Technology Abstract upload to GitHub} due Feb 26,
  2018


Please review the plagiarism and quoting guidelines chapter in the
handbook.

\WHERE{\YES}{S:plagiarism}{Week 1} 

The following is an example of how you upload your technology
abstracts to github.
\smallskip

{\hfill \href{https://github.com/cloudmesh-community/hid-sample/tree/master/technology}{Upload
    Tech Abstract to github $\mapsto$}}

\smallskip

Please direct any questions toward the TA's, additionally there is a
README available below. 

{\hfill
  \href{https://github.com/cloudmesh-community/hid-sample/blob/master/technology/README.md}{README
    $\mapsto$}}

\smallskip

The report will be generated on Mondays at 9:00 am est and made
available by 12:00 pm est of the same day.
\smallskip

\URL{https://drive.google.com/open?id=1h6_ZRmlCRIFMHG861wSyriPzn9rXxgKT}

\end{exercise}

Learning Outcome:

\begin{itemize}
\item Learn how to not plagiarize.
\item Learn how to write simple abstract.
\item Learn how to use our simple templates while just treating it as a
  form
\item Learn how to use citations
\item Learn how to upload your document to github (you will need this
  for all work from now on)

\end{itemize}

\subsection{Cloud Accounts}
\label{a:accounts}
\begin{exercise}

  {\bf Account Creation: github.com}. If you do not have a github.com
  account, go to github.com and apply for a \url{https://github.com}
  account. Write down your account name and remember the password. You
  will need the account for upcoming assignments.

\end{exercise}

\begin{exercise}

  {\bf Account Creation: futuresystems.org}. 
  
  If you do not have a
  futuresystems.org account go to
  \url{https://portal.futuresystems.org/user/register} and apply for an
  account. Write down your account name and remember the password. You
  will need the account for upcoming assignments.

\end{exercise}

\begin{exercise}
  {\bf Account Creation: chameleoncloud.org}. 
  
  If you do not have an account on chameleon cloud please go to 
  \url{https://www.chameleoncloud.org} and apply for an
  account only. Do not apply for a project. Write down your account
  name and remember the password. You will need the account for
  upcoming assignments.
\end{exercise}

\begin{exercise}
Fill out the form so we can activate the accounts for you
\url{https://goo.gl/forms/W0MdgoJoY8F6Vt9Q2}
You will need the account for
  upcoming assignments.
\end{exercise}


Learning Outcome:
\begin{itemize}
\item This class uses cloud computing resources. THe resources listed
  are the onc e we use. Most importantly you will get a github
  repository created that you will be using for the class at
  \url{https://github.com/cloudmesh-community}. We can only create
  your repository if we know your github username. Please note the
  repository is viewable. As we are working as prat of this class as
  open source project it is natural that all work is done in the open.
\end{itemize}




\subsection{Entry Survey}
\label{a:survey-entry}
\begin{exercise}
Please fill out the following survey ASAP as it will determine some of the class material we prepare based on your feedback. The survey is really simple and can be finished in under 5 minutes.

\url{https://goo.gl/forms/Q04FW9eBM7eyL0Lv1}
\end{exercise}

Learning Outcome:
\begin{itemize}
\item We ned to identify what background you have and which resources
  we have as to potentially adapt our class material, but also to
  identify which resources you need to use in addition to your own.
\end{itemize}



\subsection{Account Setup}

\begin{exercise} {\bf Github Setup} due Feb, 2018

This exercise will assist you with your github account setup. Please
find the README.yml file in the link below and copy it into your
github repository and fill out with your
information. \url{https://github.com/cloudmesh-community/hid-sample/blob/master/README.yml}
Remove the README.md file. After your \verb|.yml| file is correctly setup you
now need to put your bio in your repository, naming it like so
\verb|bio-lastname-firstname.tex| where lastname is your lastname and so
on. Put the \verb|.tex| file in your repository and use the latex function
subsection\{lastname,firstname\} followed by the text of your bio.

\end{exercise}

Learning Outcome:
\begin{itemize}
\item Setting up your github repository. Letting us know who you
  are. Filling out a form
\end{itemize}



\subsection{REST}
\label{E:REST-a}
\begin{exercise}
Read the Sections Overview of Rest Section~\ref{c:rest} and Eve~\ref{s:eve-intro}.
\end{exercise}

\begin{exercise} {\bf Develop an Eve REST Service} due Feb, 2018

See: \url{https://piazza.com/class/jbku81aeli95rz?cid=55}

In this exercise, you will be developing an Eve REST service related
to Cloud Services. We will enhance this assignment throughout the
semester once we have spoken about cloud services in more detail. At
this time you are expected to write a REST service that exposes
information form your computer, such as, processor name, RAM,
Disk. Please identify what information would be useful to have and how
to obtain that information related to your operating
system. Additionally identify how to integrate dynamic data.


\end{exercise}

Learning Outcome:
\begin{itemize}
\item Learn how to create a simple REST service. This assignment will
  be expanded upon throughout the semester. Make sure you have a solid
  understanding of it.
\end{itemize}

\begin{exercise}
Assignment catch up: please look at all previous assignments and do
them. 
\end{exercise}


Learning Outcome:
\begin{itemize}
\item It is not good in this class to procrastinate.
\end{itemize}

\begin{exercise}

AI cloud service: This assignment will evolve, therefore we just
provide the link to it in piazza. Please visit it:

\url{https://piazza.com/class/jc9dcfnbi045kv?cid=34}
\end{exercise}


Learning Outcome:
\begin{itemize}
\item You will be combining various technologies and practices that
  you learned in class as part of a project assignment.
\item You wil be exposed to various AI technologies and algorithms
  that you can explore further by integrating them in your project.
\end{itemize}



\subsection{Swagger REST Services}
\label{E:REST-swagger}
\subsubsection{PART A: Elementary Swagger Server}
 
\begin{exercise}
  This is for residential students (The swagger preparation was
  mentioned last week in class):

 \begin{itemize}

 \item (A) you are expected to have done major portions of the Swagger
   code gen assignment this includes

  \begin{itemize}
  \item (1) pick of a resource you like to implement a REST service
    for
  \item (2) read the swagger spec
  \item (3) complete the spec for the resource as much as possible
  \item (4) generate the code via swagger codegen
  \end{itemize}

  This naturally means you need swagger, python and other needed
  libraries set up on your machine. We will help refining your
  resource. We will help working with you on a simple back-end
  implementation that you will improve. You will be using github for
  all of this

\item (B) In addition to this I recommend that you

  \begin{itemize}

  \item (1) find on the network how to put Raspbian on an SD card,
    create your self an md file for this as this may be different for
    different operating systems.

 
  \item (2) download Raspbian on your machine so you can burn it as you
    will get 5 sd cards
 
    In the lab, those that finish most of the swagger service will
    switch to building a cloud cluster.
 
  \item (3) in the lab you will be first thing change the password to
    each of them before you put them on the network. Maybe there is a
    way to do this directly from your computer after you have put
    Raspbian on the sd card. But I do not know if that's possible.

 
  \item (4) figure out how to configure Raspbian without a monitor,
    while just using an SD card and your laptop, write an md file

  \end{itemize}
 \end{itemize}
 
In case of questions, lets engage in a discussion. If you have md
files for information already in your repo, please post URLs. This
assignment is about openly sharing. Naturally, you should not wait
till someone else does it, you should take leadership yourself.

 
Naturally, focus on the swagger service before starting to get more
involved with the cluster

\end{exercise}

Learning Outcome:

\begin{itemize}
\item get familiar with OpenAPI/Swagger. Use a community tool to
  generate a REST service template. 
\end{itemize}

\subsubsection{PART A: Functional Swagger Server}
\begin{exercise} {\bf Cloud and Big Data REST Service with Swagger}
  due March 5, 2018 Spec before Feb 15, 2018

\begin{itemize}
     
\item (A) Read the Chapters about REST and Swagger. This includes
  Swagger specification and Swagger codegen. There is also a video
  that introduces you to Swagger \URL{https://youtu.be/0_Ub13py_K8}
 
\item (B) Read the Document
  \URL{https://laszewski.github.io/papers/NIST.SP.1500-8-draft.pdf}
 
  We will be collaboratively developing a new version of this document
  while not using examples as in the previous document but swagger
  OpenAPI 2.0 specifications.
 
\item (C) You will pick one of the existing resources or identify a
  new resource that you would like to specify. Simply go to:

\URL{https://docs.google.com/document/d/12FUtHlEzQwxc3hjyki3RbU0iMfBrcyoSMKeq3aEDPk/edit?usp=sharing
} 

and add your name to one of the resource. Make sure there is only one
name for each resource even if you work in a team.
 
\item (D) For the resource, you chose you will be developing with
  Swagger a useful REST service related to cloud computing and Big
  Data.

 
 \end{itemize}

 TA's will provide more details as we will avoid that everyone
 develops the same service.
 
 You will use this specification create a swagger python service and
 implement functions to enable a real implementation of the service
 that is useful.

 
Examples:

\begin{itemize}
\smallskip
\item (A) develop a service to upload files to a file system with REST calls.

\item (B) start a cluster of virtual machines on a supercomputer

\item (C) start a cluster of virtual machines on OpenStack

\end{itemize}


Additional resources are listed in the instructor answer. 
\end{exercise}

Learning Outcome:

\begin{itemize}
\item Complete your first programming task, identify how to use
  swagger codegen
\item use this service throughout the class.
\end{itemize}

\subsubsection{PART B: Reproducible Swagger Services}

\begin{exercise}
It is important to be able to reproduce the Services and not just
create a code that you can run. In order to do that we will only store
the absolute minimum information in github and autogenerate the code
via the yaml file, your controller and a Makefile that you will
design.

Naturally we needed you to have understood the Swagger codegen tool
first, so you need to be familiar on how to create a swagger service.
Now that you are we can continue with this step. This will even
include removing code that you uploaded to github.

Thus, We like you to review the following and engage with us in online
meetings if this is not clear. It is actually rather simple. This is
used to prepare you for the way we expect you to deliver the final
project. It also replicates the way we have taught you to compile your
latex pdf document. Wait -- what has latex to do with swagger services?
Technically nothing conceptional we do

 
\begin{itemize}
\item make in paper dir -> produces latex document

\item make in swagger dir-> produces swagger service

\item and in future make in project dir -> produces project services
\end{itemize}
 

So we just use the same framework, which is very convenient!

Now let us get back to the swagger service generation:
 
One of the goals of this project is to create a REPRODUCIBLE framework
for generating the services. It is not enough to just develop a
swagger service. You will need to generate the service from a shell
script and/or makefile. As both technologies are available on any
computer including Windows it is your responsibility to make sure you
have make and bash installed. And use them.

Swagger-codegen applied to your yaml file will create a directory
structure. This directory structure is not to be checked into
github. Instead you will check in the makefile or shell script (bash)
that creates it.

You will see in the controller dir a number of \verb|controller_*.py|
files, you will copy them into your github dir. The names of the
controller files can be automatically specified based on the content
in your yaml file.

 

Thus you will have a directory with the following contents, replace
000 with the id you have

 
\begin{lstlisting}
hid-sp18-000/swagger/Makefile
hid-sp18-000/swagger/conroller_a.py
hid-sp18-000/swagger/conroller_b.py
\end{lstlisting}
 

We assume that swagger-codegen is a shell variable allowing you to run
swagger-codegen.

This could be different on different systems. You will be documenting
this for your system. On OSX this is trivial as you just use brew to
install and

\begin{lstlisting}
export SWAGGER-CODEGEN=swagger-codegen
\end{lstlisting}
 
will typically work. On other systems you may have to specify the jar
file. You will be using an environment variable regardless which OS
you are on

If you use a Makefile you will be defining the following tags

 
\begin{itemize}
\item make clean -- removes the code generated

\item make service -- creates the swagger service from the yaml file
  and places the controllers in the appropriate directory

\item make start  -- starts the service

\item make stop -- stops the service

\item make test -- executes a number of tests against the service
\end{itemize}

You are allowed as part of this use nose or any other unit test. 

\end{exercise}


Learning Outcome:

\begin{itemize}
\item Learn about code management, Makefiles, and create only the
  absolute needed code. 
\item Learn about the management of the code in github and avoid
  checking in all unnecessary files. This is how we expect you develop
  your project.
\item Use this service throughout the class.
\end{itemize}

\subsubsection{Swagger Container}

\begin{exercise}

STEP 3 generating Swagger REST Containers

After you have finished the STEP 2 in the Swagger series of exercises,
You will now generate a container. However, you will neither check in
the container in docker hub nor into github.

 

Instead, you will generate a container with a Dockerfile. You will add
to your makefile the tag

 
\begin{lstlisting}
make container -- which will generate the container form the
                  Dockerfile. Your container will be named cloudmesh-<YOURTOPIC>
\end{lstlisting}

Where topic is how you named your service

As we will start multiple services from multiple students you need to
have a proper namespace in the yaml specification file This may be
different for different people. For example

 
\begin{lstlisting}
cloudmesh/var/<id>
cloudmesh/aws/vm/<id>
cloudmesh/google/vm/<id>
cloudmesh/openstack/vm/<id>
cloudmesh/filter/<id>
\end{lstlisting}

 

Naturally, if you have better url suggestions please integrate,
however, we need a unique prefix for each service so if we were to
combine them we can do that.

Reply to this followup discussion
\end{exercise}


Learning Outcome:

\begin{itemize}
\item Learn how to use a simple container. 
\end{itemize}

\subsection{Technology Paper}
\label{E:616-tech-paper}

\begin{exercise} {\bf Technology Paper} due March 20, 2018

\begin{itemize}
Read the following two points to assist you in starting your paper 
\item (A) pick one of the technologies identified by you or if you see a
  hid that has already picked a technology for this assignment you can
  also pick one form that students list also

\item (B) write a LaTeX paper about the technology. Make sure not to
  plagiarize. The maximum number of quotes is about 25\% if
  needed. Please see the scientific writing section

\end{itemize}
 WE RECOMMEND YOU GET STARTED ON THIS RIGHT AWAY AS YOU ALSO WILL HAVE
 TO DO A TUTORIAL AND THE FINAL PROJECT.

Additional details can be found here:
\url{https://github.com/cloudmesh-community/hid-sample/blob/master/paper-instructions.md}

\end{exercise}


Learning Outcome:
\begin{itemize}
\item Learn how to write an extended abstract (contrasting an abstract
  in a paper).
\item Learn about LaTeX and bibtex
\item Learn how to use our trivial template nad fill it out
\item Learn how to use github for paper writing
\item Learn that because LaTeX supports structure rather than WYSIWYG
  the paper quality is usually higher than in Word
\item Learn that all images and tables and references are not counting
  towards paper length and they will be placed at the end of the paper
\item Read our LaTeX section
\item In your final project report it is likely you need to write
  similar to this report 2 pages to introduce the project

\end{itemize}


\subsection{Tutorial}
\label{E:616-tutorial}

\begin{exercise} {\bf Tutorial} due March 26, 2018

{\bf This is for online and residential students.}

Residential students with a substantial tutorial such as install a
kubernetes cluster on PI or docker swarm or some other larger topic
are exempt from this assignment. However, if you are working towards
an A+ consider adding an additional tutorial.

 
We like you to pick a technology and develop for this technology a
tutorial. The tutorial can be written either in markdown or in
LaTeX. However, when we like that you do not use enumerations for
steps that you document. Instead use sentences such as
 
First, we do 
 
Second, we need to
 
Next, we implement
 
Please use sections, subsections and so on to the
structure. Additionally please create a separate directory for images
called {\bf images} where all the images are stored like the structure
below. 

\begin{lstlisting}
hid-sp18-000/tutorial/images
hid-sp18-000/tutorial/tutorial.tex
\end{lstlisting}
 
DO NOT USE SCREENSHOTS FOR CODE EXAMPLES OR CAPTURE OF THE TERMINAL
OUTPUT.  USE ASCII and put it either in \verb|lstlisting| in latex, or in an
indented codeblock in markdown. Please do not add python, bash or
other markings to your codeblock, as we need real simple markdown if
you chose that. Make sure text in LaTeX.

 
We will make additional suggestions for tutorial topics in the
document in which we collect the tutorial list. We will also indicate
for these suggesting a maximum number of students able to work
together on that tutorial.

{\hfill
  \href{https://docs.google.com/document/d/1L2-wYc7S7hb5u6ZNtKpTvlXqKMkqq-B38hlaBCw-eww/edit?usp=sharing}{Tutorial
    List$\mapsto$}}



\end{exercise} 


Learning Outcome:
\begin{itemize}
\item Learn how to write in markdown a tutorial like text
\item Learn how to properly use indentation and verbatim 
\item Learn Markdown is not about WYSIWYG
\item In your project you will need to provide a tutorial like mark
  down section in which you explain to execute your project. This
  assignment helps you on how to do that.
\end{itemize}

\subsection{Project}
\label{E:project}

\begin{exercise} 

The policies and format about the project are discussed in Sections
\ref{s:project-format} and
\ref{s:paper-rules}.

Some project ideas for this years class are listed in Section
\ref{s:project-ideas}. Some previous projects are listed in previous
volumes that you can find in Section \ref{S:p-int}.

It is purpose of this class that you define your own project
first. There should not be an overlap between other projects. All
projects should be approved.  Two month before the project deadline
the class was informed via Piazza to think about the project and
engage in a discussion about the topic. A one page snapshot of the
project paper is due on April 2nd as posted in piazza.

The final paper is due Fri 4/27 9am. This is a hard deadline.

\end{exercise} 


Learning Outcome:
\begin{itemize}
\item Learn how to do a project
\item Learn how to avoid checking in unnecessary executables, data and
  just focus on the source you modify. THis includes not to clone or
  copy other peoples work, but to leverage downloads.
\item Learn how to write a project report
\item Learn how to do a benchmark
\item Learn how to communicate your effort in written English.
\item Leverage your experience from writing the 2 page paper
\item Leverage your experience from writing your REST service
\item Leverage your experience from writing in LaTeX and using bibtex
\item leverage your experience to write an installation section in
  markdown for your project
\item Learn how to use virtual machines or containers or both.
\end{itemize}
%\end{comment}

\section{Assignments I523}
\label{s:i523-assignment}
\index{Assignments!I523}

\subsection{Open Source Efforts}

\begin{exercise} {\bf Linux Foundation Project Summary}
Go to the Web Page of the Linux Foundation, pick a projects (the TAs
may randomly assign it for you). For the project write within
one week a 1-page summary using our Paper format. We recommend that
you use sharelatex. Please use <iuusername>@iu.edu as login. 

The link for the project is located at
\URL{https://www.linuxfoundation.org/projects/}

Make sure that you use our template. Make sure your use bibtex
resources. Focus on the Project, you only have one back to
write. Images , references, and tables to not count towards the one
page. All images will be placed at the end of the paper by our
template. You can write in a group, but each member must be lead on
one paper. You can use sharelatex to write the paper collaboratively.

Learning outcomes:
\begin{itemize}
\item Learn about open source linux projects
\item Learn how to use sharelatex
\item Learn how ideas it is to write a paper in latex while using our
  template
\item Learn how to do citations and organize them in bibtex
\end{itemize}
\end{exercise} 

\begin{exercise} {\bf Linux Foundation Project Review}
You will be assigned by the TA three papers (it can not be a paper that you
write as part of a group). You will write constructive  review to
improve the technology summary.
Learning outcomes:
\begin{itemize}
\item Learn from others 
\item Learn how to use sharelatex
\item Learn how ideas it is to write a paper in latex while using our
  template
\item Learn how to do citations and organize them in bibtex
\end{itemize}

\end{exercise} 

\begin{exercise} {\bf Linux Foundation Project Summary Improvements}
Integrate the improvement suggestions in your paper. Your paper will
have 3 reviews from students. Improve your paper.
\end{exercise} 

\begin{exercise} {\bf Linux Foundation Project Summary Extension}
Integrate the improvement suggestions in your paper. Your paper will
have 3 reviews from students. Improve your paper.
\end{exercise} 

