% chktex-file 1

\FILENAME\

\section{Basic Emacs}
\label{C:emacs}

One of the most useful short manuals for emacs is the following reference
card. It takes some time to use this card efficiently, but the most
important commands are written on it. Generations of students have
literally been just presented with this card and they learned emacs
from it.

\URL{https://www.gnu.org/software/emacs/refcards/pdf/refcard.pdf}


There is naturally also additional material available and a great
manual. You could also look at

\URL{https://www.gnu.org/software/emacs/tour/}


From the last page we have summarized the most useful and
\textbf{simple} features. And present them here. One of the hidden gems
of emacs is the ability to recreate replay able macros which we include
here also. You ought to try it and you will find that for data science
and the cleanup of data emacs (applied to smaller datasets) is a gem.

Notation

\begin{longtable}[]{@{}ll@{}}
\toprule
Key & Description\tabularnewline
\midrule
\endhead
C & Control\tabularnewline
M & Esc (meta character)\tabularnewline
\bottomrule
\end{longtable}

Here are some other ways on what to do if you have accidentally
pressed a wrong key:

\begin{itemize}
\item \verb|C-g| If you pressed a prefix key (e.g. \verb|C-x|) or you invoked
a command which is now prompting you for input (e.g. Find file:
\ldots{}), type \verb|C-g|, repeatedly if necessary, to cancel. \verb|C-g| also
cancels a long-running operation if it appears that Emacs has frozen.

\item \verb|C-/| If you executed a command and Emacs has modified your buffer, use \verb|C-/| to
undo that change. 
\end{itemize}

To save the current file say 

\begin{longtable}[]{ll}
\toprule
Key & Description\tabularnewline
\midrule
\endhead
\verb|C-x C-w| & Write the buffer to file \tabularnewline
\verb|C-x C-s| & Write the buffer to file and quit Emacs \tabularnewline
\bottomrule
\end{longtable}


Moving around in buffers can be done with cursor keys, or with the
following key combinations:

\begin{longtable}[]{ll}
\toprule
Key & Description\tabularnewline
\midrule
\endhead
\verb|C-f| & Forward one character\tabularnewline
\verb|C-n| & Next line\tabularnewline
\verb|C-b| & Back one character\tabularnewline
\verb|C-p| & Previous line\tabularnewline
\bottomrule
\end{longtable}

Here are some ways to move around in larger increments:

\begin{longtable}[]{ll}
\toprule
Key & Description\tabularnewline
\midrule
\endhead
\verb|C-a| & Beginning of line\tabularnewline
\verb|M-f| & Forward one word\tabularnewline
\verb|M-a| & Previous sentence\tabularnewline
\verb|M-v| & Previous screen\tabularnewline
\verb|M-<| & Beginning of buffer\tabularnewline
\verb|C-e| & End of line\tabularnewline
\verb|M-b| & Back one word\tabularnewline
\verb|M-e| & Next sentence\tabularnewline
\verb|C-v| & Next screen\tabularnewline
\verb|M->| & End of buffer\tabularnewline
\bottomrule
\end{longtable}

You can jump directly to a particular line number in a buffer:

\begin{longtable}[]{ll}
\toprule
Key & Description\tabularnewline
\midrule
\endhead
\verb|M-g| g & Jump to specified line\tabularnewline
\bottomrule
\end{longtable}

Searching is easy with the following commands

\begin{longtable}[]{ll}
\toprule
Key & Description\tabularnewline
\midrule
\endhead
\verb|C-s| & Incremental search forward\tabularnewline
\verb|C-r| & Incremental search backward\tabularnewline
\bottomrule
\end{longtable}

Replace

\begin{longtable}[]{ll}
\toprule
Key & Description\tabularnewline
\midrule
\endhead
\verb|M-|\% & Query replace\tabularnewline
\bottomrule
\end{longtable}

Killing (``cutting'') text

\begin{longtable}[]{ll}
\toprule
Key & Description\tabularnewline
\midrule
\endhead
\verb|C-k| & Kill line\tabularnewline
\bottomrule
\end{longtable}

Yanking

\begin{longtable}[]{ll}
\toprule
Key & Description\tabularnewline
\midrule
\endhead
\verb|C-y| & Yanks last killed text\tabularnewline
\bottomrule
\end{longtable}

Macros

Keyboard Macros

Keyboard macros are a way to remember a fixed sequence of keys for later
repetition. They're handy for automating some boring editing tasks.

\begin{longtable}[]{ll}
\toprule
Key & Description\tabularnewline
\midrule
\endhead
\verb|M-x (| & Start recording macro\tabularnewline
\verb|M-x )| & Stop recording macro\tabularnewline
\verb|M-x e| & Play back macro once\tabularnewline
\verb|M-5 C-x-e| & Play back macro 5 times\tabularnewline
\bottomrule
\end{longtable}

Modes

``Every buffer has an associated major mode, which alters certain
behaviors, key bindings, and text display in that buffer. The idea is to
customize the appearance and features available based on the contents of
the buffer.'' modes are typically activated by ending such as \verb|.py|,
\verb|.java|, \verb|.rst|, \ldots{}

\begin{longtable}[]{ll}
\toprule
Key & Description\tabularnewline
\midrule
\endhead
\verb|M-x python-mode| & Mode for editing Python files\tabularnewline
\verb|M-x auto-fill-mode| & Wraps your lines automatically when they get longer
than 70 characters.\tabularnewline
\verb|M-x flyspell-mode| & Highlights misspelled words as you type.\tabularnewline
\bottomrule
\end{longtable}


\subsection{Org Mode}

Emacs has some very advanced features that you can activate via a
mode. One such feature is to organize a TODO list via org-mode.

Instead of us designing our own video, we point to a community
tutorial such as

\video{Cloud}{18:04}{Emacs org-mode}{https://www.youtube.com/watch?v=Kde5YVUwDTQ}{Youtube}


\subsection{Programming Python with Emacs}

Emacs comes by default with syntax highlighting for python when you
edit a \verb|.py| file. This is really all you need. It also comes with a
python ide that you can use and customize.

Python auto-completion for Emacs:

  \URL{https://github.com/tkf/emacs-jedi}


Some more information is available at

\URL{https://realpython.com/blog/python/emacs-the-best-python-editor/}
\URL{https://www.emacswiki.org/emacs/PythonProgrammingInEmacs}

\subsection{Emacs Keys in a Terminal}

One of the real great features of knowing emacs is that you can set
all your editors to emacs shortcuts. This includes pyCharm, but also
bash. IN bash you simply say 

\begin{verbatim}
set -o emacs
\end{verbatim}

in your bash prompt. Additionally, if you do not have a window systems
configured, you can run emacs directly in the terminal with 

\begin{verbatim}
emacs -nw
\end{verbatim}

This you can log in to a remote computer and if it has emacs
installed. Use it in the terminal. This would replace editors such as
vi, vim, nano, pico or others that work in a terminal.

\subsection{\LaTeX~and Emacs}

LaTeX is directly supported by emacs and nothing has to be
changed. However, a collection of information about additional
\LaTeX~features for emacs is available at

\URL{https://www.emacswiki.org/emacs/LaTeX}

Of interest are for example also 

* \verb|M-x flyspell-mode|: allowing to do spell checking in the window
* predictive mode: https://www.emacswiki.org/emacs/PredictiveMode
* preview latex
* whizzy tex

However instead of previes and whizzy tex we recommend to use
\url{https://www.emacswiki.org/emacs/LatexMk} which comes pre-installed
and allows you to do editing in one terminal, while previewing the
update on change in another window.

LatexMk is all integrated in our report Makefiles and the Book format,
so you will be able to use this immediately. This is similar to share
latex, but much faster and without collaborators editing the same file.