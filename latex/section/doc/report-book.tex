% chktex-file 8

\FILENAME\

\section{Writing a Scientific Article or Conference Paper}
\label{S:writing}
\index{Writing}

An important part of any scientific research is to document it. This
is often done through scientific conferences or journal
articles. Hence it is important to learn how to prepare and submit
such papers. Most conferences accept typically the papers in PDF
format but require the papers to be prepared on MSWord or in
\LaTeX. While working with many students in the past we noticed
however that those students using Word often spend unnecessarily
countless hours on trying to make there papers beautiful while
actually violating the template provided by the
conference. Furthermore, we noticed that the same students had issues
with bibliography management. Instead of Word helping the student it
provided the illusion to be easier than \LaTeX~but when adding up the
time spend on the paper we found that \LaTeX~actually saved time. This
has been especially true with the advent of collaborative editing
services such as sharelatex~\cite{www-sharelatex} and
overleaf~\cite{www-overleaf}.

In this section we provide you with a professional template that is used
based on the ACM standard that you can use to write
papers. Naturally this will be extremely useful if the quality of your
research is strong enough to be submitted to a conference. We structure
this section as follows. Although we do not recommend that you use
MSWord for your editing of a scientific paper, we have included a short
section about it and outline some of its pitfalls that initially you may
not think is problematic, but has proven to be an issue with students.
Next we will focus on introducing you to \LaTeX~and showcasing you the
advantages and disadvantages. We will dedicate an entire section on
bibliography management and teach you how to use jabref which clearly
has advantages for us.

Having a uniform report format not only helps the students but allows
instructors to integrate the comparison of paper length and effort as
part of teaching a course. We have added an entire section to this
chapter that discusses how we can manage a \emph{Class Proceedings}
from papers that are contributed by teams in the class.

\subsection{Professional Paper Format}\label{professional-paper-format}

The report format we suggest here is based on the standard ACM
proceedings format. It is of very high quality and can be adapted for
your own activities. Moreover, it is possible to use most of the text to
adapt to other formats in case the conference you intend to submit your
paper to has a different format. The ACM format is always a good start.

Important is that you do not need to change the template but you can
change some parameters in case you are not submitting the paper to a
conference but use it for class papers. Certainly you should not change
the spacing or the layout and instead focus on writing content. As for
bibliography management we recommend you use jabref which we will
introduce in Section~\ref{S:bibliographies}.

We recommend that you carefully study the requirements for the report
format. We would nat want that your paper gets rejected by a journal,
conference or the class just because you try to modify the format or
do not follow the established publication guidelines. The template we
are providing is available from:

% \URL{https://github.com/bigdata-i523/sample-hid000/tree/master/paper1}

\URL{https://github.com/cloudmesh-community/hid-sample/tree/master/paper}

You will find in it a modified ACM proceedings templates that you must
use. 

\subsection{Submission Requirements}\label{submission-requirements}

Although the initial requirement for some conferences or journals is the
document PDF, in many cases you must be prepared to provide the source
when submitting to the conference. This includes the submission of the
original images in an images folder. You may be asked to package the
document into a folder with all of its sources and submit to the
conference for professional publication.

\subsection{Microsoft Word vs. \LaTeX}\label{microsoft-word}

Microsoft word will provide you with the initial impression that you
will safe lots of time writing in it while you see the layout of the
document. This will be initially true, but once you progress to the
more challenging parts and later pages such as image menagement and
bibliography management you will see some issues. This include that
figure placement in Word need sto be done just right in order for images
to be where they need. We have seen students spending hours with the
placement of figures in a paper but when they did additional changes the
images jumped around and were not at the place where teh students
expected them to be. So if you work with images, make sure you
understand how to place them. Also always use relative caption counters
so that if an image gets placed elsewhere the counter stays consistent.
So nefer use just the number, but a reference to the figure when referring
to it. Recently a new bibliography management system was added to Word.
However, however it is not well documented and the references are placed
in the system bibliography rather than a local managed bibliography.
This mah have severe consequences when working with many authors on a
paper. The same is true when using Endnote. We have heard in many
occasions that the combination of endnote and Word destroyed documents.
You certainly do not want that to happen the day before your deadline.
Also in classes we observed that those using LaTeX deliver better
structured and written papers as the focus is on text and not beautiful
layout.

For all these reasons we do not recommend that you use Word.

In LaTeX where we have an easier time with this as we can just ignore
all of these issues due to relative good image placement and excellent
support for academic reference management. Hence, it is in your best
interest to use LaTeX. The information we provide here will make it easy
for you to get started and write a paper in no time as it is just like
filling out a form.

\subsection{Working in a Team}\label{working-in-a-team}

Today research is done in potentially large research teams. This also
include the writing of a document. There are multiple ways this is done
these days and depends on the system you chose.

In MSWord you can use onedrive, while for LaTeX you can use sharelatex
and overleaf. However, in many cases the use of github is possible as
the same groups that develop the code are also familiar with github.
Thus we provide you here also with the introduction on how to write a
document in github while group members can contribute.

Here are the options:

\begin{description}

\item [LaTeX and git:] This option will likely safe you time as you can use
  jabref also for managing collaborative bibliographies and
\item [sharelatex:] an online tool to write latex documents
\item [overleaf:] an online tool to write latex documents
\item [MS onedrive:] It allows you to edit a word document in collaboration.
  We recommend that you use a local installed version of Word and do the
  editing with that, rather than using the online version. The online
  editor has some bugs. See also (untested):
  \url{http://www.paulkiddie.com/2009/07/jabref-exports-to-word-2007-xml/},
  \url{http://usefulcodes.blogspot.com/2015/01/using-jabref-to-import-bib-to-microsoft.html}
\item [Google Drive:] google drive could be used to collaborate on text that
  is than pasted into document. However it is just a starting point as
  it does not support typically the format required by the publisher.
  Hence at one point you need to switch to one of the other systems.
\end{description}

\subsection{Time Management}\label{timemanagement}

Obviously writing a paper takes time and you need to car-fully make sure
you devote enough time to it. The important part is that the paper
should not be an after thought but should be the initial activity to
conduct and execute your research. Remember that

\begin{enumerate}

\item  It takes time to read the information
\item  It takes time understand the information
\item  It takes time to do the research

\end{enumerate}

For deadlines the following will get you in trouble:

\begin{enumerate}

\item
  \emph{There are still 10 weeks left till the deadline, so let me start
  in 4 week \ldots{}}. Procrastination is your worst enemy.
\item
  If you work in a team that has time management issues address them
  immediately
\item
  Do not underestimate the time it takes to prepare the final submission
  into the submission system. Prepare automated scripts that can deliver
  the package for submission in minutes rather than hours by hand.
\end{enumerate}

\subsection{Paper and Report Checklist}\label{paper-checklist}
\index{Writing!Checklist}

In this section we summarize a number of checks that you may perform to
make sure your paper is properly formatted and in excellent shape.
Naturally this list is just a partial list and if you find things we
should add here, let us know.

One good way is to either copy the checklist into a file or print out
just this pages and check with a pen if the particular issue occurs.

%A checklist with a subset of these issues that you can add to your draft is available at

%\URL{https://github.com/bigdata-i523/sample-hid000/blob/master/paper1/issues.tex}

\subsection{Content}

\begin{itemize}[label=$\Box$]
    \item Is the paper formal paper and not an experience report? 
    \item Do not include phrases such as ``In week 1 we did this''
    \item When writing the \textit{proposal} do not use the word ``proposal''
      write the document as if it would be teh final paper. We see too
      many reports at the end forgetting to remove the word proposal
      in the final paper, so we can not tell if you did it or if it is
      still a proposal. As the final paper is not a proposal we reject
      such papers and you get a 1/3 grade reduction. To avoid this,
      just do not use the word proposal.
    \item When writing the abstract do not make it a
      proposal. Abstracts are no proposals. Avoid phrases such as We
      propose to do, We intend to show and so on. If the paper intends
      to show things you are still in the draft phase of the
      paper. However, if you say We show, that would be good. Let us
      just assume you intended to show something but did not achieve
      then you can say ``We intended to show this but we it was not
      possible to verify. We have provided reasons for this in the
      paper''. As you can see not only the intention is communicated,
      but the result. If you just focus on the intent that is just a
      proposal and is not a proper abstract.
    \item Add keywords to the paper, where the first two are your HID,
      and your class number.
    \item If your paper is an introduction or overview paper, please
      do not assume the reader to be an expert. Provide enough
      material for the paper to be useful for an introduction into the
      topic.
    \item If your paper limit is x number of pages but you want to
      hand in x plus 100 pages. If however you page limit is 2 pages
      and you hand in 4 or 6 pages that is no issue.
\end{itemize}

\subsection{Submission}

\begin{itemize}[label=$\Box$]
    \item Do not make changes to your paper during grading, when your
      repository should be frozen.
     \item Do not use filenames and directory names that have spaces
       in them only use [a-z0-9]*
     \item Make all file names lower case other than Makefile and
       README.yml
     \item You are required to run yamllint README.yml on all team
       members README.yml including your own. All of them must pass. Do this
       on the first day you start writing the paper. Only push and
       commit the files when they pass this test. If you do not have
       yamllint you can write one in python. Its 3 lines of code.
     \item Have you included the paper in the submission system (In
       our case git). This includes all images, bibliography files and
       other material that is needed to build the paper from scratch?
     \item Have you made sure your paper compiles with \textit{make} and
       the provided Makefile before you committed?
     \item Are all images checked in?
     \item Did you submit the report.bib file?
\end{itemize}

\subsection{Bibliography}

\begin{itemize}[label=$\Box$]
  \item Are you managing your references in jabref and endnote (we need
    both)
  \item In the author field, authors are separated with an \textit{and}
    and not a comma.
  \item The filename for the bibliography is report.bib.
  \item Bibtex labels must have any spaces, \_ or \& in it
  \item Fix citations in text that show as [?]. This means either your
    report.bib is not up-to-date or there is a spelling error in the
    label of the item you want to cite, either in report.bib or in
    report.tex
  \item Urls in citations are never placed in howpublished, instead we
    use url = \{ \}. \verb|howpublished| is just used for a text sting such
    as Web Page, Blog, Repository and others like that. Do not use
    just the word Web, as it could be a Web Site or a Web Page. You
    need to be specific.
  \item Do not use the \verb|\url={ ]| in teh text, instead use a
      citation.
    \item Are you references correct? References to a paper are no
      afterthought, they should be properly cited. Use jabref and make
      sure the citation type of the reference is correct and fill out
      as many fields as you can. Some journals and conferences have
      for example special requirements that go beyond the requirements
      of for example jabref. One example is that many conferences
      require you that wne you cite papers form another conference to
      augment the conferences not only with the location where the
      conference took place, but also with the dates the conference
      took place. Unfortunately, this is information that is often
      only available through additional google queries and many
      reference entries you find in the internet do not have this
      information readily available.
\end{itemize}

\subsection{Writing}

\begin{itemize}[label=$\Box$]
    \item Have you spellchecked the paper?
    \item Have you grammar chacked the paper?
    \item Use proper capitaliztion in the title, see: \url{https://capitalizemytitle.com/}
     \item Are you using \textit{a} and \textit{the} properly?
    \item Short form of verbs is for spoken language. Do not use them
      in scientific writing. Example: can't is incorrect, cannot is correct.
    \item Do not use phrases such as \textit{shown in the Figure
        below}. Instead, use \textit{as shown in Figure 3}, when
      referring to the 3rd figure, but use the \textit{ref} \textit{label}
      macros.
    \item Do not use the word \textit{I} instead use \textit{we} even if you
      are the sole author. In many cases you may want to avoid using
      the word \textit{we} also.
    \item Do not use the phrase \textit{In this paper/report we show}
      instead use \textit{We show}. It is not important if this is a
      paper or a report and does not need to be mentioned. 
    \item If you want to say \textit{and} do not use \textit{\&} but use the
      word \textit{and}.
    \item Use a space after \verb|. , :|
    \item When using a section command, the section title is not
      written in all-caps as the \LaTeX~template will do this
      automatically for you. Thus it is \verb|\section{Introduction}|
      and NOT \verb|\section{INTRODUCTION}|.

\end{itemize}

\subsection{Citation Issues and Plagiarism}

\begin{itemize}[label=$\Box$]
   \item It is your responsibility to make sure no plagiarism occurs. 
   \item When stating claims you added the proper citations. 
   \item Do avoid paraphrasing long quotations (whole sentences or
     longer) form other papers.
   \item Double check your paper if you have quote from other papers
     and included the citation.
   \item The \verb|\cite{}| command must not be in the beginning of
     the sentence or paragraph, but in the end, before the period
     mark. Example: \ldots a library called Message Passing
     Interface (MPI) [7]. 
   \item Put a space between the citation mark and the previous word
     or better use  \verb|~|.
   \item There must not be any citation in the abstract or conclusion.
   \item Citations cannot be included in section headings they need to be
     included in the text. 
\end{itemize}

\subsection{Character Errors}

The following errors are very often found and must be avoided.

\begin{itemize}[label=$\Box$]
    \item To emphasize a word, use \textit{emphasize} and not
      ``quote''. Quotes are reserved for quotes from other papers and
      must not be used to emphasize words or phrases.
      to put around a word that you like to emphasize.
    \item Generally we do not us {\bf bold fett} text. Instead use
      \textit{em}.
    \item Erroneous use of quotation marks, i.e.\ use \verb|``quotes''|,
       but not the double quote that you find on your keyboard such as
       \verb|" "|.
    \item When using the characters \& \# \% \_ put a backslash before
      them so that they show up correctly.
    \item Pasting and copying from the Web often results in non-ASCII
      characters to be used in your text, please remove them and
      replace accordingly. This is the case for quotes, dashes and all
      the other special characters.
    \item If you see a f\hspace{-0.05cm}igure and not a figure in text
      you copied from a text that has the fi combined as a single
      character. It happens often with combinations of f such as fi fl
      ff
\end{itemize}

\subsection{Structural Issues}

\begin{itemize}[label=$\Box$]
    \item Does your paper include an Acknowledgement section.
    \item Is the acknowledgment including all the people appropriately
      that helped you in your activity. 
    \item In case of a class and if you do a multi-author paper, you
      need to add an appendix called \textit{Workbreak Down} describing
      who did what in the paper,after the bibliography
    \item Do you fullfill the minimum page length such as defined in
      the submission guideline. Remember that 
      images, tables and figures do not count towards the page length.
    \item Do not artificially inflate your paper if you are below the
      page limit.
    \item In case you have an appendix it is included after the
      bibliography
\end{itemize}

\subsection{Figures and Tables}

\begin{itemize}[label=$\Box$]
  \item Images must be at least 300dpi if they are not in a scalable
    format such as PDF which you can generate from Powerpoint and
    other drawing programs. 
  \item If you use Microsoft products, use ppt 4:3 ratio for drawing
    concet images. In case there is a powerpoint in the submission,
    the image must be exported as PDF.
  \item If you have OSX, you are allowed to use omnigraffle.
  \item Make sure you capitalize Figure 1, Table 2 when used in a
    sentence.
  \item Do use \verb|\label{}| and \verb|\ref{}| to automatically create
    figure numbers.
  \item Figure caption must be bellow the image.
  \item Table captions must be above the image.
  \item Do not include the titles of the figures in the figure itself
    but instead use the caption or that information.
  \item All images must be in native format, e.g. \verb|.graffle|, \verb|.pptx|,
    \verb|.png|, \verb|.jpg| in the images directory
  \item Do not submit eps images. Instead, convert them to PDF
  \item The image files must be in a single directory named \textit{images}.
  \item Make the figures large enough so we can read the details. If
    needed make the figure over two columns
  \item Do not worry about the figure placement if they are at a
    different location than you think. Figures are allowed to
    float. To illustrate this case we force all images to be placed at
    the end of the paper, although you may have included it at a
    special location in the paper. This forces you to avoid the
    phrases as seen in teh following image, but you need to use the
    ref and label features in LaTeX.
  \item In case you copied a figure from another paper you need to ask
    for copyright permission. In case of a class paper, you must
    include a reference to the original in the caption. In general we
    like to avoid this for the reports and like that you produce
    original pictures.
  \item Remove any figure that is not referred to explicitly in the
    text with a ref command. Again just putting in the number will not
    be good enough. This allows you to place the figure in the final
    submission at a location without needing to fix the numbers.
  \item Do not use textwidth as a parameter for includegraphics, but
    instead use \verb|\columnwidth| as demonstrated in our template. 
  \item Figures should be reasonably sized and often you just need to
    add columnwidth e.g.

       \verb|/includegraphics[width=1.0\columnwidth]{images/myimage.pdf}|

    Do not play with the size, just leave it with 1.0.

\end{itemize}


If you observe something missing let us know.

\subsection{Example Paper}\label{example-paper}

An example report in PDF format is available:

%\begin{itemize}

%\item
%  \href{https://github.com/cloudmesh/classes/blob/master/docs/source/format/report/latex/report.pdf}{report.pdf}
%\end{itemize}

\URL{https://github.com/cloudmesh-community/hid-sample/blob/master/paper-instructions.pdf}

\subsection{Creating the PDF from LaTeX on your
Computer}\label{creating-the-pdf-from-latex-on-your-computer}

Latex can be easily installed on any computer as long as you have
enough space. Furthermore if your machine can execute the make command
we have provided in the standard report format a simple

%\href{https://github.com/cloudmesh/classes/blob/master/docs/source/format/report/latex/Makefile}{Makefile}

\href{https://github.com/cloudmesh-community/hid-sample/blob/master/paper/Makefile}{Makefile}

that allows you to do editing with immediate preview as documented in
the LaTeX lesson.

\subsection{Draft: Class Specific README.md}\label{class-specific-readme.md}

For the class we will manage all papers via github.com. You will be
added to our github at

\URL{https://github.com/cloudmesh-community}

Previously we used

\URL{https://github.com/bigdata-i523}

and assigned an hid (homework index directory) directory with a unique
hid number for you. In addition, once you decide for a project, you will
also get a project id (pid) and a directory in which you place the
projects. Projects must not be placed in hid directories as they are
treated differently and a class proceedings is automatically created
based on your submission.

As part of the hid directory, you will need to create a README.md file
in it, that \textbf{must} follow a specific format. The good news is
that we have developed an easy template that with common sense you can
modify easily. The template is located at

\URL{https://raw.githubusercontent.com/bigdata-i523/sample-hid000/master/README.md}

As the format may have been updated over time it does not hurt to
revisit it and compare with your README.md and make corrections. It is
important that you follow the format and not eliminate the lines with
the three quotes. The text in the quotes is actually yaml. Yaml is a
data format the any data scientist must know. If you do not, you can
look it up. However, if you follow our rules you should be good. If you
find a rule missing for our purpose, let us know. We like to keep it
simple and want you to fill out the \emph{template} with your
information.

Simple rules:

\begin{itemize}

\item
  replace the hid number with your hid number.
\item naturally if you see sample- in the directory name you need to

  delete that as your directory name does not have sample- in it.
\item do not ignore where the author is to be placed, it is in a list
  starting with a \verb|-|

\item there is always a space after a \verb|-|

\item do not introduce empty lines

\item do not use TAB and make sure your editor does not bay accident
  automatically creates tabs. This is probably the most frequent error
  we see.

\item do not use any \verb|: & _| in the attribute text including titles

\item an object defined in the README.md must have on a single type
  field.  For example in the project section. Make sure you select
  only one type and delete the other

\item in case you have long paragraphs you can use the \textgreater{}
  after the abstract

\item Once you understood how the README.md works, please delete the
  comment section.
\item Add a chapter topic that your paper belongs to

\end{itemize}

\subsection{Exercises} \bigskip


\begin{exercise}
\label{E:Report.1}
Install latex and jabref on your system
\end{exercise}

\begin{exercise}
\label{E:Report.2}
Check out the report example directory. Create a PDF and view it. Modify
and recompile.
\end{exercise}

\begin{exercise}
\label{E:Report.4}
Learn about the different bibliographic entry formats in bibtex
\end{exercise}

\begin{exercise}
\label{E:Report.5}
What is an article in a magazine? Is it really an Article or a Misc?
\end{exercise}

\begin{exercise}
\label{E:Report.6}
What is an InProceedings and how does it differ from Conference?
\end{exercise}

\begin{exercise}
\label{E:Report.7}
What is a Misc?
\end{exercise}

\begin{exercise}
\label{E:Report.8}
Why are spaces, underscores in directory names problematic and why should you avoid using them for your projects
\end{exercise}

\begin{exercise}
\label{E:Report.9}
Write an objective report about the advantages and disadvantages of programs to write reports.
\end{exercise}

\begin{exercise} 
\label{E:Report.10}
Why is it advantageous that directories are lowercase have no underscore or space in the name?
\end{exercise}

