\section{Overview}\label{C:overview-doc}

\FILENAME\

Using a paper as part of your project planing is an important learning
outcome. Instead of starting with a project we recommend that you
start with a paper to direct your research.

This argument is made also by the following presentation.

\video{Writing}{57:39}{How to write a paper by Simon Peyton Jones}{https://www.youtube.com/watch?v=VK51E3gHENc}

We do recommend that you read the sections in this part carefully as they will introduce you to important tools that make writing a paper relatively simple while allowing professional paper format and bibliography management tools.

\begin{WARNING} 

Partially outdated:

To get a first impression we have also prepared a number of videos that may help you. However, note that the format for papers used in these videos is different from the class and you must use the written documentation instead and use that format. Paper not using our format will be returned without review. I suggest you start right from the beginning.

The videos that show you the ACM paper template that we do not use but
give a good introduction of share latex and jabref

\video{i524}{8:49}{ShareLaTeX }{https://youtu.be/PfhSOjuQk8Y}

\video{i524}{14:41}{jabref}{https://youtu.be/cMtYOHCHZ3k}

\end{WARNING}

The best way to start with a document is our template at 

\URL{https://github.com/cloudmesh-community/hid-sample/tree/master/paper}

\begin{exercise}\label{E:Documentation.1}
Watch the three lectures about How to write a paper, ShareLaTeX, and jabref.
\end{exercise}