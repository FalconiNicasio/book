\FILENAME\

\section{Acknowledgements}
\label{S:acknowledgements}

In many cases you not only want but must to include an acknowledgement
section. In some cases you may be tempted to eliminate this section as
you think you are out of space and the acknowledgement section may
give you some additional space. This however is the wrong strategy and
you should not do this. Instead you should shorten your paper
elsewhere and leave enough space for acknowledgements.

In some cases where you get financial support from a university or a
funding agency for a project such as from NIH or NSF specific
information {\bf must} be included. The best way is to verify with
your coauthors. Additional acknowledgements may have to be added and
you need te evaluate if for example significant help on the paper or
the work that lead up to the paper warrants co-authorship.

An issue that we have seen often is for example when a professor has
helped significantly on the paper but is not properly acknowledged.
This can even lead to the professor asking you to remove him from the
acknowledgement. A bad acknowledgement example is the following:

\begin{quote} 

  We like to thank Professor Zweistein for his help in
  compiling the \LaTeX~paper. 

\end{quote}

We do not think that the professor will be happy with this
acknowledgement as it sounds like the only thing that was provided was
the help on \LaTeX that you should have done anyways without the help
of the professor. Ask yourself, if he introduced you to the field, has
helped you with preparing the text, has given you insights, has
corrected things in your paper, made suggestions. So instead of the
above maybe a more general term such as \textit{helped with the paper}
would be more appropriate. If not leaving it off is more appropriate.
In some cases you may wan to invite your professor to become a
co-author. In some cases you may want to even include this handbook as
a citation.
