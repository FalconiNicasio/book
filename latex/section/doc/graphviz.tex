\MDNAME\
%%%%%%%%%%%%%%%%%%%%%%%%%%%%%%%%%%%%%%%%%%%%%%%%%%%%%%%%%%%%%%%%%%%%%%%%%%%%%%%
% DO NOT MODIFY THIS FILE
%%%%%%%%%%%%%%%%%%%%%%%%%%%%%%%%%%%%%%%%%%%%%%%%%%%%%%%%%%%%%%%%%%%%%%%%%%%%%%%

\section{Graphviz}

\begin{itemize}
\item
  \url{https://graphviz.gitlab.io/resources/}
\end{itemize}

\subsection{Installation}

\subsubsection{OSX}

On OSX you can install graphviz with

\begin{lstlisting}
brew install graphviz
\end{lstlisting}

On OSX there is also a graphviz version available that includes a GUI.
THe link to this software is:

\begin{itemize}
\item
  \url{http://www.pixelglow.com/graphviz/}
\end{itemize}

It can be downloaded from

\begin{itemize}
\item
  \url{http://www.pixelglow.com/downloads/graphviz-1.13-v16.dmg}
\end{itemize}

There is also an additional tool that is distributed by the community
that is called doteditor and can be installed with

\begin{lstlisting}
brew cask install doteditor
\end{lstlisting}

If you have issues with brew cask install, you can also install it by
hand while going to

\begin{itemize}
\item
  \url{https://vincenthee.github.io/DotEditor/}
\end{itemize}

Online versions of graphviz are also available, but we have not tested
them

\begin{itemize}
\item
  \url{http://www.webgraphviz.com/}
\item
  https://dreampuf.github.io/GraphvizOnline/
\item
  http://viz-js.com/
\item
  http://graphviz.it/\#/gallery/unix.gv
\end{itemize}

There are many more

\subsection{Usage}

To use Graphviz create a dot and run the following command.

\begin{lstlisting}
dot -Tpng filename.dot -o filename.png
\end{lstlisting}

This will create a png file. Other formats are also possible such as
svg, or PDF

\begin{lstlisting}
dot -Tsvg filename.dot -o filename.svg
dot -TPDF filename.dot -o filename.pdf
\end{lstlisting}

For inclusion in latex documents we recommend you create PDF output as
it has a much better quality and is smaller in size than png.

\subsection{The Dot Format}

Put an example dot file here

\subsection{Exercise}

\begin{exercise}
Develop a REST service that takes a graph as input and returns a rendered version of the graph in a specified format. Make sure you can pass the format as a parameter.
\end{exercise}

\begin{exercise}
Develop a REST service that takes a graph as input and returns a URL of the rendered graph while storing the output onto a data server. The data server is another rest service, from which the result can be picked up. 
\end{exercise}

\begin{exercise}
For IU students. Develop a REST service that takes a graph as input and returns a URL of the rendered graph while storing the output onto a data server. The data server is another rest service, from which the result can be picked up. Use box and/or google drive that are offered by IU as services. Make sure not to expose your passwords or access keys
\end{exercise}

